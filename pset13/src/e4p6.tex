Using what you have learned from the examples above, write a program that meets the following requirements:

\begin{itemize}
    \item The program commands the robot to drive a designated distance at a chosen speed, and stops when that distance is reached.
    \item If the bump sensor is triggered, the robot reverses direction and backs up for 5 seconds then stops.
\end{itemize}

\begin{solution} \
\begin{lstlisting}
function [] = driveUntilBumpThenRunAway(speed, distance)
    pub = rospublisher('/raw_vel');
    sub_bump = rossubscriber('/bump');
    msg = rosmessage(pub);

    % get the robot moving
    msg.Data = [speed, speed];
    send(pub, msg);

    start = rostime('now');
    
    reverseFlag = 0;
    reverseStart = 0;
    
    while 1
        current = rostime('now');
        elapsed = current - start;
        if (elapsed.seconds > distance/speed) && (reverseFlag == 0) % Here we are saying the if the elapsed time is greater than 
            %distance/speed, we have reached our desired distance and we should stop

            message.Data = [0,0]; % set wheel velocities to zero if we have reached the desire distance
            send(pubvel, message); % send new wheel velocities
            break %leave this loop once we have reached the stopping time
        end
        
        % wait for the next bump message
        bumpMessage = receive(sub_bump);
        % check if any of the bump sensors are set to 1 (meaning triggered)
        if any(bumpMessage.Data)
            reverseFlag = 1;
            reverseStart = rostime('now');
        end
        
        reverseElapsed = current - reverseStart;
        if (reverseElapsed.seconds > 5) && (reverseFlag == 1)
            message.Data = [0, 0];
            send(pubvel, message);
            break
        end
    end
end
\end{lstlisting}
\end{solution}