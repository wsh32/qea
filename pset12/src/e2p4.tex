Determine (by hand) the unit binormal vector.

\begin{solution}
\begin{align*}
    \boldsymbol{\hat{B}} &= \boldsymbol{\hat{T}} \times \boldsymbol{\hat{N}} \\
    &= \frac{b\sin u}{\sqrt{a^2+b^2}} \hat{\boldsymbol{i}} + \frac{b\sin u}{\sqrt{a^2+b^2}} \hat{\boldsymbol{j}} + \frac{a}{\sqrt{a^2+b^2}} \hat{\boldsymbol{k}}
\end{align*}
\end{solution}